\documentclass{article}
\usepackage[margin=3cm]{geometry}
\usepackage{amsmath}

\setlength{\parindent}{0pt}
\setlength{\parskip}{1em}

\begin{document}

\begin{center}
  \large{\textbf{Editorial: Unforgivable Curse (hard version) - 1800E2}}\\
  \vspace{0.2em}
  \large{Avighna}\\
  \vspace{0.2em}
  \large{June 2024}
\end{center}

Note that $s_i$ denotes the $i^{\text{th}}$ character of $s$, but \texttt{s.substr(x, y)} denotes the substring starting from index $x$ (that, is, $s_{x+1}$) with a length of $y$. The former is one-indexed while the latter is zero-indexed.

We can swap $s_i$ with $s_{i+k}$ and with $s_{i+k+1}$, so when $n \le k$., we can't perform any swaps and we need $s=t$ from the beginning.

In a substring ($x$) of length $k+2$, we can perform the following swaps:
\section{Swap Sequence $1$}
$[x_1, x_2, ..., x_{k+1}, x_{k+2}]$, swap $x_1$ and $x_{k+1}$\\
$\rightarrow [x_{k+1}, x_2, ..., x_1, x_{k+2}]$, swap $x_{k+1}$ and $x_{k+2}$ (note that these two characters \textbf{are} at a distance of $k+1$ from each other)\\
$\rightarrow [x_{k+2}, x_2, ..., x_1, x_{k+1}]$, swap $x_1$ and $x_{k+2}$\\
$\rightarrow [x_1, x_2, ..., x_{k+2}, x_{k+1}]$\\

to swap $s_{k+1}$ with $s_{k+2}$. To swap $s_1$ with $s_2$, we can do this:
\section{Swap Sequence $2$}
$[x_1, x_2, ..., x_{k+1}, x_{k+2}]$, swap $x_2$ and $x_{k+2}$\\
$\rightarrow [x_1, x_{k+2}, ..., x_{k+1}, x_2]$, swap $x_1$ and $x_2$\\
$\rightarrow [x_2, x_{k+2}, ..., x_{k+1}, x_1]$, swap $x_1$ and $x_{k+2}$\\
$\rightarrow [x_2, x_1, ..., x_{k+1}, x_{k+2}]$\\

which is really the same thing as swap sequence 1, but mirrored.

Thus, in $s$, using swap sequence $1$, we can swap $s_{k+1}$ with $s_{k+2}$ by considering \texttt{x = s.substr(0, k + 2)}. Similarly, we can swap $s_{k+2}$ with $s_{k+3}$ by considering \texttt{x = s.substr(1, k + 2)}. If we keep doing this, we arrive at the conclusion that we can swap $s_i$ with $s_{i+1}$ $\forall i \in [k+1, n-1]$. Since we can swap all adjacent characters ahead of $s_i$, we can shuffle these characters however we want.

By considering substrings of length $k+2$ starting from the back (that is, \texttt{s.substr(n-k-2, n-1)}, etc.), we'll be able to swap $s_i$ with $s_{i+1}$ $\forall i \in [1, n-k-1]$

If these two intersect, that is, if $s_{i+1}$ for $i=n-k-1$ overlaps with $s_i$ for $i = k+1$, then we'll be able to swap any two characters of the entire string. Let's calculate when this happens:

$(n-k-1)+1\ge k+1$\\
$\rightarrow n-k\ge k+1$\\
$\rightarrow n\ge 2k+1$ or $n > 2k$\\

So if $n>2k$, we can just sort $s$ and $t$ and check for equality. If they're equal after sorting, we can transform $s$ to $t$, otherwise we can't.

What if $n=2k$? We can swap any two characters among the first $k$ characters and the last $k$ characters, but can we swap $s_k$ and $s_{k+1}$? If we could find a way to do this, we'd be able to swap any two characters of our string and we could then perform the same 'sort and check for equality' maneuver.

Our string is currently $[s_1, ..., s_k, s_{k+1}, ..., s_n]$. Let's perform the following swaps:
\section{Swap Sequence $3$}
$[s_1, s_2, ..., s_{k-1}, s_k, s_{k+1}, s_{k+2}, ..., s_n]$, swap $s_1$ and $s_{k+1}$\\
$\rightarrow [s_{k+1}, s_2, ..., s_{k-1}, s_k, s_1, s_{k+2}, ..., s_n]$, move $s_{k+1}$ to be ahead $s_k$ (we can do this since we can swap any two characters in the first half)\\
$\rightarrow [s_2, ..., s_{k-1}, s_k, s_{k+1},s_1, s_{k+2}, ..., s_n]$, move $s_k$ to the beginning of the string\\
$\rightarrow [s_k, s_2, ..., s_{k-1}, s_{k+1},s_1, s_{k+2}, ..., s_n]$, swap $s_k$ with $s_1$ (again, they're at a distance of $k+1$)\\
$\rightarrow [s_1, s_2, ..., s_{k-1}, s_{k+1},s_k, s_{k+2}, ..., s_n]$\\

Using this swap sequence, we've successfully swapped $s_k$ with $s_{k+1}$. Here's a concrete example that uses swap sequence $3$: $n=6, k=3, s=\text{abcdef}$:
$\text{abc def}$\\
$\rightarrow\text{dbc aef}$\\
$\rightarrow\text{bcd aef}$\\
$\rightarrow\text{cbd aef}$\\
$\rightarrow\text{abd cef}$\\


For $n<2k$, any character $s_i$ $\forall i \in [n-k+1, k]$ cannot be swapped with any other character since they're at a distance of less than $k$ from every other character (it is sufficient to check the first and last characters to prove this. Go ahead and try, you'll end up getting four inequalities). So \texttt{s.substr(n - k, 2k - n)} must equal \texttt{t.substr(n - k, 2k - n)}.

This solution has a time complexity of $\mathcal{O}(n)$.
\section{Code}
\begin{verbatim}
std::string get(bool b) {
    return b ? "YES" : "NO";
}

void solve() {
    ll n, k;
    std::string s, t;
    std::cin >> n >> k >> s >> t;

    if (n <= k) {
        std::cout << get(s == t) << "\n";
    } else {
        bool ans = true;
        if (2 * k - n > 0) {
            ans = s.substr(n - k, 2 * k - n) == t.substr(n - k, 2 * k - n);
        }
        std::sort(s.begin(), s.end());
        std::sort(t.begin(), t.end());
        std::cout << get(ans && s == t) << "\n";
    }
}
\end{verbatim}

\end{document}