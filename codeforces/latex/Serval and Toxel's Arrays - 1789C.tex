\documentclass{article}
\usepackage[margin=3cm]{geometry}
\usepackage{amsmath}
\usepackage{hyperref}

\setlength{\parindent}{0pt}
\setlength{\parskip}{1em}

\begin{document}

\begin{center}
  \large{\textbf{Editorial: Serval and Toxel's Arrays - 1789C}}\\
  \vspace{0.2em}
  \large{Avighna}\\
  \vspace{0.2em}
  \large{June 2024}
\end{center}

We're asked to find the sum of the values of all pairs $(i,j)$ where $0\le i<j\le m$. Let's rewrite this as the sum of the values of all pairs with $i=0$ plus the sum of the values of all pairs with $i=1$, and so on till $i=m-1$.

If we're able to find a way to calculate the first sum, and then a way to quickly transition between the sum when $i=x$ and the sum when $i=x+1$, then we can add all these sums to obtain our answer.

Let the sum of the values of all pairs $(i, j)$ with $i<j \le m$ for some fixed value of $i$ be denoted by $s_i$. As stated previously, our answer is $\displaystyle \sum_{i=0}^{m-1} s_i$. $s_i$ itself is the sum of the values of the pairs $(i, i+1), (i, i+2), ..., (i, m)$. Let $v_{(a,b)}$ denote the value of the pair $(a,b)$. Now let's create an array called $x_i$ where its $j^\text{th}$ element (denoted by $x_{i,j}$) is equal to $v_{(i,i+j)}$ ($x_i = [v_{(i, i+1)}, v_{(i, i+2)}, ..., v_{(i, m)}]$). Let's insert the element $n$ behind the first element of this array: the first element is $x_{i,1}$, so $x_{i,0} = n$. Now let's find this array's adjacent difference, call this $\Delta_i$. $\Delta_{i,j} = x_{i,j}-x_{i,j-1}$ $\forall j \in [1, m-i]$.

Here's where all these definitions become useful. $s_i=\displaystyle \sum_{j=1}^{m-1} x_{i,j} = \sum_{j=1}^{m-1} \sum_{k=0}^{j} \Delta_{i,k}$ (by definition), so $\Delta_{i,j}$ represents the change in the number of unique values when the array formed by the concatenation of $A_i$ and $A_{j-1}$ morphs to the array formed by the concatenation of $A_i$ and $A_j$.

Described ahead is an algorithm that computes $\Delta_{0,j}$. Imagine there being 'slots' which you use to swap out one number for another when going from $A_0$ to $A_1$, or from $A_1$ to $A_2$, for example. We say that a slot is unused if we haven't already used it to swap a number.

\begin{quote}
$A_0 = [1,0,3,4,5] \rightarrow A_1 = [1,6,3,4,5]$
\end{quote}

In the above example, we've replaced a $0$ with a $6$ via slot $2$, which was previously unused. If we then did

\begin{quote}
$A_1=[1,6,3,4,5] \rightarrow A_2=[1,9,3,4,5]$
\end{quote}

then we'd have replaced a $6$ with a $9$ via the already used slot $2$.

To calculate the change in the number of unique elements when the array we're concatenating $A_0$ with goes from $A_0$ to $A_1$, then $A_1$ to $A_2$, then $A_2$ to $A_3$ and so on; in this case, from $A_0$ to $A_i$ we need to notice that if we swap out $a$ to $b$ and $b$ isn't present in $A_0$, we add another element to the concatenation. So we add $1$ to $\Delta_{0,i}$. If we add this element from a used slot, this means that we'd also added an element previously that we're now removing, so we need to make sure that we subtract $1$ from $\Delta_{0,j}$ if this element was also absent from $A_0$.

Let's now digress and look at what happens when you find the prefix sum of the prefix sum of an array (since $s_i$ is defined this way in terms of $\Delta_i$)

\begin{quote}
$[a,b,c,d]$ is the original array\\
$[a,a+b,a+b+c,a+b+c+d]$ is the first prefix sum\\
$[a,2a+b,3a+2b+c,4a+3b+2c+d]$ is the second prefix sum\\
\end{quote}

Using this notion, adding $x$ to the concatenation of $A_0$ and $A_i$ can be thought of as $x$ contributing a value of $m-i+1$ to $s_i$ when it is absent from $A_0$. Thus $s_i$ is equal to the sum of $mn$ and the contributions of all numbers from $0$ to $m+n$ (why are we adding $mn$? Just adding the contributions of all numbers doesn't account for the fact that $\Delta_{0,0}=n$ (actually, $\Delta_{i,0}=n$ $\forall i \in [0,m-1]$). $\Delta_i$ has $m+1$ elements, and we're summing up $x_i[1...m]$, not $x_i[0...m]$, so we're subtracting $n$ from the sum of $x_i[0...m]=m(n+1)$).

Now that we've found $s_0$, let's use its value to find $s_1$, then use $s_1$'s value to find $s_2$ and so on, adding them along the way. To do this, we have to find a way to (implicitly) obtain $\Delta_i$ from $\Delta_{i-1}$ (in reality we're just adding and subtracting things from $s_0$) and then repeat the 'second prefix sum' summing procedure to subtract and add values to $s_0$.

Here's an example of a transition:
\subsection{Test case}
\begin{verbatim}
10 10
4 6 9 12 16 20 2 10 19 7
1 3
5 4
2 17
2 18
6 11
7 1
8 17
5 5
5 5
2 2
\end{verbatim}

$$
\Delta_0 = \left[
\begin{array}{ll}
10, \\
1, & \text{{Insert 3 in slot 1}} \\
0, & \text{{Insert 4 in slot 5}} \\
1, & \text{{Insert 17 in slot 2}} \\
-1+1, & \text{{Remove 17 from slot 2, insert 18 in slot 2}} \\
1, & \text{{Insert 11 in slot 6}} \\
1, & \text{{Insert 1 in slot 7}} \\
1, & \text{{Insert 17 in slot 8}} \\
-0+1, & \text{{Remove 4 from slot 5, insert 5 in slot 5}} \\
-1+1, & \text{{Remove 5 from slot 5, insert 5 in slot 5}} \\
-1+0 & \text{{Remove 18 from slot 2, insert 2 in slot 2}} \\
\end{array}
\right]
$$

We implicitly convert this to $\Delta_1$ by removing the first element, adjusting the second element to be $n$, and subtracting $m-i+1$ from the contribution of the element added (adding for the element removed) in the second row, since in $\Delta_1$, our original array is no longer $A_0$, it's $A_1$ (so it's no longer contributing, it's literally part of the array). Now adjust $s_0$ by subtracting the contributions of the two elements (or one, if you're not removing anything) that change during the transition if they're present, update it so that the value you just replaced (in our case, we replaced $4$ at the first position of our array with $3$; we went from $[4,6,9,12,16,20,2,10,19,7]$ to $[3,6,9,12,16,20,2,10,19,7]$) is now marked as absent and the value you replaced it with is marked as present, and then add back the contributions of these two elements if they're present. Doing this will successfully give you $s_1$.

$$
\Delta_1 = \left[
\begin{array}{ll}
10, \\
1, & \text{{Insert 4 in slot 5}} \\
1, & \text{{Insert 17 in slot 2}} \\
-1+1, & \text{{Remove 17 from slot 2, insert 18 in slot 2}} \\
1, & \text{{Insert 11 in slot 6}} \\
1, & \text{{Insert 1 in slot 7}} \\
1, & \text{{Insert 17 in slot 8}} \\
-1+1, & \text{{Remove 4 from slot 5, insert 5 in slot 5}} \\
-1+1, & \text{{Remove 5 from slot 5, insert 5 in slot 5}} \\
-1+0 & \text{{Remove 18 from slot 2, insert 2 in slot 2}} \\
\end{array}
\right]
$$

Note that the real $\Delta_1$ array is never found by our algorithm (although the sum of all elements of its prefix sum excluding the first is ($s_i$)), I've provided it here solely for reference (the algorithm wouldn't be fast enough if it actually found $\Delta_1$).

\section{Code}
\href{https://codeforces.com/contest/1789/submission/255600704}{Submission}

\end{document}