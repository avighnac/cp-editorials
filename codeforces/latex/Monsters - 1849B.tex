\documentclass{article}
\usepackage[margin=3cm]{geometry}
\usepackage{amsmath}

\setlength{\parindent}{0pt}
\setlength{\parskip}{1em}

\begin{document}

\begin{center}
  \large{\textbf{Editorial: Monsters - 1849B}}\\
  \vspace{0.2em}
  \large{Avighna}\\
  \vspace{0.2em}
  \large{June 2024}
\end{center}

\section{Initial (Naïve) Approach}

The most obvious initial approach would be to simply implement what the question is asking. Keep track of the number of killed monsters, keep finding the greatest health monster, subtracting $k$ from it and updating the value of the counter variable when necessary.

The code for this (which will be too slow) is as follows:

\begin{verbatim}
#include <bits/stdc++.h>
 
using namespace std;
 
#define ll long long
 
void solve() {
  ll n, damage;
  cin >> n >> damage;
  vector<ll> m(n);
  for (ll i = 0; i < n; ++i) {
    cin >> m[i];
  }
 
  ll killed = 0;
  while (killed != n) {
    ll mind = max_element(m.begin(), m.end()) - m.begin();
    if (m[mind] - damage <= 0) {
      cout << mind + 1 << " ";
      killed++;
    }
    m[mind] -= damage;
  }
  cout << "\n";
}
 
int main() {
  ll t;
  cin >> t;
  while (t--) {
    solve();
  }
}
\end{verbatim}

\section{Slightly Improved Naïve Approach}

At this point, one may notice that instead of finding the largest monster in $O(n)$, one can instead use a \texttt{std::priority\_queue} to do the same in $O(\log{n})$, but this is still too slow to pass.

There are various more optimizations that can be further made to the naïve approach, but I highly doubt anything will get it fast enough to be accepted. That being said, let's now move on to a better method.

\section{Optimal Solution}

Key Idea: If the initial health of the monsters is given as $[a_1, a_2, a_3, ..., nk, a_m, a_{m+1}, a_{m+2}, ...]$, and $a_x\mod k\neq0$, then the monster $nk$ will be the first one to die (i.e. get reduced to 0).

Here is the proof for this.

Note that if $nk > a_x$ at every stage, then this is trivially true (since we use the special attack on the largest element at every stage).

If not, then $a_x < nk$ (equality is impossible due to our indivisibility assumption). Since all monster healths are integers, $n \in Z^+$. If $n = 0$, then $nk$ has already been reduced to 0 (which is what we wanted to prove). Otherwise, $a_x > nk$ and thus $a_x > k$ ($n$ has now been proven to be a natural number) and $a_x - nk > 0$. This tells us that at every single stage, the \textit{reduced value} of any monster whose health is not divisible by $k$ can never go under $1$.

More specifically, after complete reduction, the health of any monster not divisible by $k$ will fall in the range of $[1, k)$ (if this were not true, then $nk$ would be reduced to $(n-1)k$ and then $k$ would be subtracted from $a_x$  again).

Thus, we can conclude that all monsters divisible by $k$ will be reduced first. Try to see why they will be equal to $k$ at the same time as well. Of course, since we choose the highest index first, we will go from left to right while eliminating them.

\section{Building on this idea}

We have previously concluded that all monsters divisible by $k$ will be killed first. In this process, all other monsters will also be reduced to the range $[1,k)$.

Then, since we choose monsters with the maximum health first, we'll kill the ones which are the greatest in the range $[1, k)$. These numbers have been reduced to this range as, previously, $k$ has been repeatedly subtracted from them. In other words (I'm sure some of you see what has to be done already): we kill the ones whose modulus with $k$ is the greatest first (here I'm referring to the original healths).

$a_x$ will be killed before $a_y$ if $m_x \mod k > m_y \mod k$.

I'd recommend that you try coding the solution yourself now. If you didn't completely understand any part, let your brain passively think about this idea for a few minutes and re-read this. With that being said, here's the solution code:

\begin{verbatim}
#include <bits/stdc++.h>
 
using namespace std;
 
#define ll long long
 
void solve() {
  ll n, k;
  cin >> n >> k;
  vector<pair<ll, ll>> m(n);
  for (ll i = 0; i < n; ++i) {
      cin >> m[i].first;
      m[i].first %= k;
      if (m[i].first == 0) {
        m[i].first = k;
      }
      m[i].second = -1 * i;
  }
 
  sort(m.begin(), m.end(), greater<pair<ll,ll>>());
  for (ll i = 0; i < n; ++i) {
    cout << -1*m[i].second+1<<" ";
  }
 
  cout<<"\n";
}
 
int main() {
  ll t;
  cin >> t;
  while (t--) {
    solve();
  }
}
\end{verbatim}

\end{document}